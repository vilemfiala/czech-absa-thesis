\chapter*{Úvod}\addcontentsline{toc}{chapter}{Úvod}\markboth{Úvod}{Úvod}


Analýza médií a textů je díky rychlému rozvoji umělé inteligence podstatně jednodušší než dříve. Firmy již nemusí zaměstnávat početné týmy pracovníků, kteří by osobně četli každý článek a na základě subjektivních pocitů jej hodnotili. Takové řešení navíc často přináší nekonzistentní výsledky, jelikož každý člověk může k hodnocení přistupovat odlišně. Proto se v současnosti stále více organizací obrací k umělé inteligenci, která -- po svém natrénování -- poskytuje konzistentní hodnocení.

Ne vždy je přitom nutné provádět detailní rozbor každého textu, mnohdy postačí obecná informace o celkovém sentimentu článku či recenze. Pro firmy nabízející široké portfolio produktů bývá důležité zejména to, jaký podíl recenzí je pozitivních a jaký negativních, což jim poskytuje rychlý přehled o vnímání jejich zboží ze strany zákazníků.

Problém však nastává, pokud jedna recenze či článek obsahuje kladné i záporné hodnocení. Například u notebooku si může autor recenze pochvalovat kvalitní obrazovku a dlouhou výdrž baterie, ale současně kritizovat nedostatečnou odolnost klávesnice. V takové situaci je vhodnější zaměřit se na konkrétní aspekty či témata a zjišťovat, jaký postoj k nim text zaujímá.

Cílený sentiment proto představuje užitečnou alternativu, kdy se text analyzuje s ohledem na zvolený aspekt a výsledkem je hodnocení vztahující se k tomuto aspektu. Tento postup umožňuje rychle identifikovat názor na vybrané aspekty a efektivně tak třídit či porovnávat texty podle konkrétních kritérií.

Zároveň však dnes neustále vznikají nové modely a metody, což klade nároky na uživatele, kteří musejí testovat různé přístupy a hledat nejvhodnější řešení pro svůj konkrétní problém. Analýza sentimentu v českém jazyce je navíc v porovnání s angličtinou stále poměrně málo rozšířená, což komplikuje výběr vhodného modelu a metod pro české texty a recenze.

\section*{Cíle}
\phantomsection
\addcontentsline{toc}{section}{Cíle}

Cílem této bakalářské práce je prozkoumat současné metody aspektové analýzy sentimentu v textu a otestovat vybrané postupy na různých typech dat v různých jazycích. Teoretická část bude zaměřena na vymezení problematiky zpracování přirozeného jazyka, analýzy sentimentu a popis vybraných metod i postupů. V praktické části budou vybrané metody aplikovány na data v češtině a v angličtině, vzájemně porovnány a vyhodnoceny z hlediska vhodnosti pro daný jazyk a konkrétní téma. Smyslem je najít či doporučit nejvhodnější metody pro práci s českými texty v kontextu mediálních textů.

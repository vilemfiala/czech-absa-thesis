\chapter*{Závěr}
\addcontentsline{toc}{chapter}{Závěr}
\markboth{Závěr}{Závěr}

Tato bakalářská práce se soustředila na aspektovou analýzu sentimentu a přispěla k jejímu rozvoji zejména v českém jazyce a mediální doméně. Úvodní kapitola představila základní principy zpracování přirozeného jazyka a popsala klíčové jazykové modely, které tvoří teoretické východisko pro všechny pozdější části práce.

Následující kapitola shrnula historický vývoj oboru, ukázala posun od statistických přístupů k moderním transformátorovým architekturám a podrobně popsala jednotlivé podúlohy ABSA i navazující složené úlohy.Samostatná třetí kapitola rozšířila perspektivu o český jazyk, zmapovala stávající výzkum a poukázala na specifické výzvy vyplývající z menší dostupnosti kvalitních datasetů.

Čtvrtá kapitola podrobně popsala veškerá použitá data, metody a modely. Pozornost byla věnována i postupu trénování jednotlivých modelů, definici vyhodnocovacích metrik a návrhu praktických testů. Po rozsáhlém experimentálním srovnání několika desítek modelů napříč doménami byla identifikována řešení, která poskytují nejlepší výsledky pro jednotlivé jazyky a domény. Tyto poznatky byly syntetizovány do souboru doporučení, který usnadní volbu vhodného modelu pro české mediální texty.

Z hlediska naplnění cílů bylo vytvořeno komplexní shrnutí dosavadních přístupů k analýze sentimentu a dílčím úlohám ABSA, provedeno experimentální srovnání vybraných modelů napříč jazyky a doménami a -- na základě těchto výsledků -- doporučeny konkrétní postupy a modely pro praktickou práci s českými texty. Tím práce splnila všechny vytyčené cíle a poskytla podklad pro další výzkum i nasazení ABSA v českém jazykovém prostředí.